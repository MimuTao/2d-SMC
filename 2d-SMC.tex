
%% bare_conf.tex
%% V1.4b
%% 2015/08/26
%% by Michael Shell
%% See:
%% http://www.michaelshell.org/
%% for current contact information.
%%
%% This is a skeleton file demonstrating the use of IEEEtran.cls
%% (requires IEEEtran.cls version 1.8b or later) with an IEEE
%% conference paper.
%%
%% Support sites:
%% http://www.michaelshell.org/tex/ieeetran/
%% http://www.ctan.org/pkg/ieeetran
%% and
%% http://www.ieee.org/

%%*************************************************************************
%% Legal Notice:
%% This code is offered as-is without any warranty either expressed or
%% implied; without even the implied warranty of MERCHANTABILITY or
%% FITNESS FOR A PARTICULAR PURPOSE! 
%% User assumes all risk.
%% In no event shall the IEEE or any contributor to this code be liable for
%% any damages or losses, including, but not limited to, incidental,
%% consequential, or any other damages, resulting from the use or misuse
%% of any information contained here.
%%
%% All comments are the opinions of their respective authors and are not
%% necessarily endorsed by the IEEE.
%%
%% This work is distributed under the LaTeX Project Public License (LPPL)
%% ( http://www.latex-project.org/ ) version 1.3, and may be freely used,
%% distributed and modified. A copy of the LPPL, version 1.3, is included
%% in the base LaTeX documentation of all distributions of LaTeX released
%% 2003/12/01 or later.
%% Retain all contribution notices and credits.
%% ** Modified files should be clearly indicated as such, including  **
%% ** renaming them and changing author support contact information. **
%%*************************************************************************


% *** Authors should verify (and, if needed, correct) their LaTeX system  ***
% *** with the testflow diagnostic prior to trusting their LaTeX platform ***
% *** with production work. The IEEE's font choices and paper sizes can   ***
% *** trigger bugs that do not appear when using other class files.       ***                          ***
% The testflow support page is at:
% http://www.michaelshell.org/tex/testflow/



\documentclass[conference]{IEEEtran}
\usepackage{mathrsfs}
\usepackage{amsmath}
\usepackage{amsthm}
\usepackage{amssymb}
\usepackage{amsfonts}
\usepackage{graphicx}
\usepackage{subfigure}
\usepackage{bm}
\usepackage{diagbox} 
\usepackage[colorlinks,linkcolor=blue]{hyperref}
\newtheorem{remark}{Remark}
\newtheorem{theorem}{Theorem}
\newtheorem{definition}{Definition} 
\newtheorem{assumption}{Assumption}






% *** CITATION PACKAGES ***
%
%\usepackage{cite}
% cite.sty was written by Donald Arseneau
% V1.6 and later of IEEEtran pre-defines the format of the cite.sty package
% \cite{} output to follow that of the IEEE. Loading the cite package will
% result in citation numbers being automatically sorted and properly
% "compressed/ranged". e.g., [1], [9], [2], [7], [5], [6] without using
% cite.sty will become [1], [2], [5]--[7], [9] using cite.sty. cite.sty's
% \cite will automatically add leading space, if needed. Use cite.sty's
% noadjust option (cite.sty V3.8 and later) if you want to turn this off
% such as if a citation ever needs to be enclosed in parenthesis.
% cite.sty is already installed on most LaTeX systems. Be sure and use
% version 5.0 (2009-03-20) and later if using hyperref.sty.
% The latest version can be obtained at:
% http://www.ctan.org/pkg/cite
% The documentation is contained in the cite.sty file itself.






% *** GRAPHICS RELATED PACKAGES ***
%
\ifCLASSINFOpdf
  % \usepackage[pdftex]{graphicx}
  % declare the path(s) where your graphic files are
  % \graphicspath{{../pdf/}{../jpeg/}}
  % and their extensions so you won't have to specify these with
  % every instance of \includegraphics
  % \DeclareGraphicsExtensions{.pdf,.jpeg,.png}
\else
  % or other class option (dvipsone, dvipdf, if not using dvips). graphicx
  % will default to the driver specified in the system graphics.cfg if no
  % driver is specified.
  % \usepackage[dvips]{graphicx}
  % declare the path(s) where your graphic files are
  % \graphicspath{{../eps/}}
  % and their extensions so you won't have to specify these with
  % every instance of \includegraphics
  % \DeclareGraphicsExtensions{.eps}
\fi
% graphicx was written by David Carlisle and Sebastian Rahtz. It is
% required if you want graphics, photos, etc. graphicx.sty is already
% installed on most LaTeX systems. The latest version and documentation
% can be obtained at: 
% http://www.ctan.org/pkg/graphicx
% Another good source of documentation is "Using Imported Graphics in
% LaTeX2e" by Keith Reckdahl which can be found at:
% http://www.ctan.org/pkg/epslatex
%
% latex, and pdflatex in dvi mode, support graphics in encapsulated
% postscript (.eps) format. pdflatex in pdf mode supports graphics
% in .pdf, .jpeg, .png and .mps (metapost) formats. Users should ensure
% that all non-photo figures use a vector format (.eps, .pdf, .mps) and
% not a bitmapped formats (.jpeg, .png). The IEEE frowns on bitmapped formats
% which can result in "jaggedy"/blurry rendering of lines and letters as
% well as large increases in file sizes.
%
% You can find documentation about the pdfTeX application at:
% http://www.tug.org/applications/pdftex





% *** MATH PACKAGES ***
%
%\usepackage{amsmath}
% A popular package from the American Mathematical Society that provides
% many useful and powerful commands for dealing with mathematics.
%
% Note that the amsmath package sets \interdisplaylinepenalty to 10000
% thus preventing page breaks from occurring within multiline equations. Use:
%\interdisplaylinepenalty=2500
% after loading amsmath to restore such page breaks as IEEEtran.cls normally
% does. amsmath.sty is already installed on most LaTeX systems. The latest
% version and documentation can be obtained at:
% http://www.ctan.org/pkg/amsmath





% *** SPECIALIZED LIST PACKAGES ***
%
%\usepackage{algorithmic}
% algorithmic.sty was written by Peter Williams and Rogerio Brito.
% This package provides an algorithmic environment fo describing algorithms.
% You can use the algorithmic environment in-text or within a figure
% environment to provide for a floating algorithm. Do NOT use the algorithm
% floating environment provided by algorithm.sty (by the same authors) or
% algorithm2e.sty (by Christophe Fiorio) as the IEEE does not use dedicated
% algorithm float types and packages that provide these will not provide
% correct IEEE style captions. The latest version and documentation of
% algorithmic.sty can be obtained at:
% http://www.ctan.org/pkg/algorithms
% Also of interest may be the (relatively newer and more customizable)
% algorithmicx.sty package by Szasz Janos:
% http://www.ctan.org/pkg/algorithmicx




% *** ALIGNMENT PACKAGES ***
%
%\usepackage{array}
% Frank Mittelbach's and David Carlisle's array.sty patches and improves
% the standard LaTeX2e array and tabular environments to provide better
% appearance and additional user controls. As the default LaTeX2e table
% generation code is lacking to the point of almost being broken with
% respect to the quality of the end results, all users are strongly
% advised to use an enhanced (at the very least that provided by array.sty)
% set of table tools. array.sty is already installed on most systems. The
% latest version and documentation can be obtained at:
% http://www.ctan.org/pkg/array


% IEEEtran contains the IEEEeqnarray family of commands that can be used to
% generate multiline equations as well as matrices, tables, etc., of high
% quality.




% *** SUBFIGURE PACKAGES ***
%\ifCLASSOPTIONcompsoc
%  \usepackage[caption=false,font=normalsize,labelfont=sf,textfont=sf]{subfig}
%\else
%  \usepackage[caption=false,font=footnotesize]{subfig}
%\fi
% subfig.sty, written by Steven Douglas Cochran, is the modern replacement
% for subfigure.sty, the latter of which is no longer maintained and is
% incompatible with some LaTeX packages including fixltx2e. However,
% subfig.sty requires and automatically loads Axel Sommerfeldt's caption.sty
% which will override IEEEtran.cls' handling of captions and this will result
% in non-IEEE style figure/table captions. To prevent this problem, be sure
% and invoke subfig.sty's "caption=false" package option (available since
% subfig.sty version 1.3, 2005/06/28) as this is will preserve IEEEtran.cls
% handling of captions.
% Note that the Computer Society format requires a larger sans serif font
% than the serif footnote size font used in traditional IEEE formatting
% and thus the need to invoke different subfig.sty package options depending
% on whether compsoc mode has been enabled.
%
% The latest version and documentation of subfig.sty can be obtained at:
% http://www.ctan.org/pkg/subfig




% *** FLOAT PACKAGES ***
%
%\usepackage{fixltx2e}
% fixltx2e, the successor to the earlier fix2col.sty, was written by
% Frank Mittelbach and David Carlisle. This package corrects a few problems
% in the LaTeX2e kernel, the most notable of which is that in current
% LaTeX2e releases, the ordering of single and double column floats is not
% guaranteed to be preserved. Thus, an unpatched LaTeX2e can allow a
% single column figure to be placed prior to an earlier double column
% figure.
% Be aware that LaTeX2e kernels dated 2015 and later have fixltx2e.sty's
% corrections already built into the system in which case a warning will
% be issued if an attempt is made to load fixltx2e.sty as it is no longer
% needed.
% The latest version and documentation can be found at:
% http://www.ctan.org/pkg/fixltx2e


%\usepackage{stfloats}
% stfloats.sty was written by Sigitas Tolusis. This package gives LaTeX2e
% the ability to do double column floats at the bottom of the page as well
% as the top. (e.g., "\begin{figure*}[!b]" is not normally possible in
% LaTeX2e). It also provides a command:
%\fnbelowfloat
% to enable the placement of footnotes below bottom floats (the standard
% LaTeX2e kernel puts them above bottom floats). This is an invasive package
% which rewrites many portions of the LaTeX2e float routines. It may not work
% with other packages that modify the LaTeX2e float routines. The latest
% version and documentation can be obtained at:
% http://www.ctan.org/pkg/stfloats
% Do not use the stfloats baselinefloat ability as the IEEE does not allow
% \baselineskip to stretch. Authors submitting work to the IEEE should note
% that the IEEE rarely uses double column equations and that authors should try
% to avoid such use. Do not be tempted to use the cuted.sty or midfloat.sty
% packages (also by Sigitas Tolusis) as the IEEE does not format its papers in
% such ways.
% Do not attempt to use stfloats with fixltx2e as they are incompatible.
% Instead, use Morten Hogholm'a dblfloatfix which combines the features
% of both fixltx2e and stfloats:
%
% \usepackage{dblfloatfix}
% The latest version can be found at:
% http://www.ctan.org/pkg/dblfloatfix




% *** PDF, URL AND HYPERLINK PACKAGES ***
%
%\usepackage{url}
% url.sty was written by Donald Arseneau. It provides better support for
% handling and breaking URLs. url.sty is already installed on most LaTeX
% systems. The latest version and documentation can be obtained at:
% http://www.ctan.org/pkg/url
% Basically, \url{my_url_here}.




% *** Do not adjust lengths that control margins, column widths, etc. ***
% *** Do not use packages that alter fonts (such as pslatex).         ***
% There should be no need to do such things with IEEEtran.cls V1.6 and later.
% (Unless specifically asked to do so by the journal or conference you plan
% to submit to, of course. )


% correct bad hyphenation here
\hyphenation{op-tical net-works semi-conduc-tor}


\begin{document}
%
% paper title
% Titles are generally capitalized except for words such as a, an, and, as,
% at, but, by, for, in, nor, of, on, or, the, to and up, which are usually
% not capitalized unless they are the first or last word of the title.
% Linebreaks \\ can be used within to get better formatting as desired.
% Do not put math or special symbols in the title.
\title{Asynchronous Siding Mode Control of Two-dimensional Markov Jump Systems in Roesser Model}


% author names and affiliations
% use a multiple column layout for up to three different
% affiliations
%\author{\IEEEauthorblockN{Michael Shell}
%\IEEEauthorblockA{School of Electrical and\\Computer Engineering\\
%Georgia Institute of Technology\\
%Atlanta, Georgia 30332--0250\\
%Email: http://www.michaelshell.org/contact.html}
%\and
%\IEEEauthorblockN{Homer Simpson}
%\IEEEauthorblockA{Twentieth Century Fox\\
%Springfield, USA\\
%Email: homer@thesimpsons.com}
%\and
%\IEEEauthorblockN{James Kirk\\ and Montgomery Scott}
%\IEEEauthorblockA{Starfleet Academy\\===
%San Francisco, California 96678--2391\\
%Telephone: (800) 555--1212\\
%Fax: (888) 555--1212}}

% conference papers do not typically use \thanks and this command
% is locked out in conference mode. If really needed, such as for
% the acknowledgment of grants, issue a \IEEEoverridecommandlockouts
% after \documentclass

% for over three affiliations, or if they all won't fit within the width
% of the page, use this alternative format:
% 
%\author{\IEEEauthorblockN{Michael Shell\IEEEauthorrefmark{1},
%Homer Simpson\IEEEauthorrefmark{2},
%James Kirk\IEEEauthorrefmark{3}, 
%Montgomery Scott\IEEEauthorrefmark{3} and
%Eldon Tyrell\IEEEauthorrefmark{4}}
%\IEEEauthorblockA{\IEEEauthorrefmark{1}School of Electrical and Computer Engineering\\
%Georgia Institute of Technology,
%Atlanta, Georgia 30332--0250\\ Email: see http://www.michaelshell.org/contact.html}
%\IEEEauthorblockA{\IEEEauthorrefmark{2}Twentieth Century Fox, Springfield, USA\\
%Email: homer@thesimpsons.com}
%\IEEEauthorblockA{\IEEEauthorrefmark{3}Starfleet Academy, San Francisco, California 96678-2391\\
%Telephone: (800) 555--1212, Fax: (888) 555--1212}
%\IEEEauthorblockA{\IEEEauthorrefmark{4}Tyrell Inc., 123 Replicant Street, Los Angeles, California 90210--4321}}




% use for special paper notices
%\IEEEspecialpapernotice{(Invited Paper)}




% make the title area
\maketitle

% As a general rule, do not put math, special symbols or citations
% in the abstract
\begin{abstract}
	abstract
\end{abstract}

% no keywords

\begin{IEEEkeywords}
	Markov jump systems, 2D systems, Siding mode control, Hidden Markov model
\end{IEEEkeywords}



% For peer review papers, you can put extra information on the cover
% page as needed:
% \ifCLASSOPTIONpeerreview
% \begin{center} \bfseries EDICS Category: 3-BBND \end{center}
% \fi
%
% For peerreview papers, this IEEEtran command inserts a page break and
% creates the second title. It will be ignored for other modes.
\IEEEpeerreviewmaketitle
  
   

\section{Introduction}
	This part is introduciton.


\section{Preliminaries} \label{priliminaries}
	In this paper, we consider the following two-dimensional Markov jump systems in Roesser model:
	\begin{equation} \label{system-equation}
	\left\{
		\begin{array}{lr}
			\begin{split}
				\emph{\textbf{x(i, j)}} &= A_{r(i,j)}x(i,j)+E_{r(i,j)}w(i,j)\\
										&+B_{r(i,j)}[(u(i,j)+f(x(i,j),r(i,j))]
			\end{split}\\
			\begin{split}
				y(i,j) = C_{r(i,j)}x(i,j)+D_{r(i,j)}w(i,j)
			\end{split}
		\end{array}
	\right.
	\end{equation}
	where
	\begin{equation*}
		\emph{\textbf{x(i, j)}} = \begin{bmatrix}
			x^{h}(i+1,j)\\
			x^{v}(i,j+1)
		\end{bmatrix}, \ 
		x(i, j) = \begin{bmatrix}
		x^{h}(i,j)\\
		x^{v}(i,j)
		\end{bmatrix}          
	\end{equation*}
	$x^{h}(i,h)\in \mathbb{R}^{n_h}$ and $x^{v}(i,h)\in \mathbb{R}^{n_v}$ represent horizontal and vertical states respectively, $u(i,j) \in \mathbb{R}^{n_u}$ and $y(i,j) \in \mathbb{R}^{n_y}$ represent the controlled input and output respectively, and $w(i,j) \in \mathbb{R}^{n_w}$ represents the exogenous disturbance which belongs to $\ell_{2}\{[0,\infty),[0,\infty)\}$. $A_{r(i,j)}$,$B_{r(i,j)}$,$C_{r(i,j)}$,$D_{r(i,j)}$ and $E_{r(i,j)}$ represent the time-varying system matrices, all of which are real known constant matrices with appropriate dimensions. Besides, we assume that the matrix $B_{r(i,j)}$ is full column rank for each $r(i,j)\in\mathcal{N}_{1}$, that is, rank($B_{r(i,j)}$)$=n_u$. The nonlinear function $f(x(i,j),r(i,j))$ satisfying the following property:
	\begin{equation}\label{nonlinear-func}
		\|f(x(i,j),r(i,j)\| \leq \delta_{r(i,j)}\|x(i,j)\|
	\end{equation}
	where $\delta_{r(i,j)}$ is a known scalar, $\|\cdot\|$ denotes the Euclidean norm of a vector. The parameter $r(i,j)$ takes values in a finite set $\mathcal{N}_{1}=\{1,2...,N_{1} \}$ with transition probability matrix $\varLambda = \{\lambda_{k\tau}\}$, and the related transition probability from mode $k$ to mode $\tau$ is given by 
	\begin{equation}
		\begin{split}
			&\Pr\{r(i+1,j)=\tau|r(i,j)=k\}\\
		    =&\Pr\{r(i,j+1)=\tau|r(i,j)=k\}=\lambda_{k\tau},\  \forall k,\tau \in \mathcal{N}_{1}
		\end{split}
	\end{equation}
	where $\lambda_{k\tau}\in[0,1]$, for all $k, \tau\in\mathcal{N}_{1}$, and $\sum_{\tau=1}^{N_1}\lambda_{k\tau}=1$ for every mode $k$.
	
	We define the boundary condition ($X_{0},\Gamma_{0}$) of system \eqref{system-equation}, as follows:
	\begin{equation} \label{boundary-condition}
	\left\{
		\begin{array}{lr}
			\begin{split}
				X_{0} &= \{x^{h}(0,j),x^{v}(i,0)|\ i,j = 0,1,2...\}\\
				\varGamma_{0} &= \{r(0,j), r(i,0)|\ i,j = 0,1,2... \}
			\end{split}
		\end{array}
	\right.
	\end{equation}
	And the corresponding zero boundary condition is assumed as $x^{h}(0,j) =0, x^{v}(i,0)=0, i,j =0,1,2...$. Besides, we further impose following assumption on $X_{0}$.
	
	\begin{assumption}\label{boundary-assumptin}
	 	The boundary condition $X_{0}$ satisfies:
	 	\begin{equation}
	 		\lim\limits_{L\to\infty}\mathbb{E}\Big\{\sum_{\ell=1}^{L}(\|x^{h}(0,\ell)\|^{2}+ \|x^{v}(\ell,0)\|^{2})\Big\} < \infty
	 	\end{equation}
	 	where $\mathbb{E}\{\cdot\}$ stands for mathematical expectation.
	\end{assumption}
	
	In practical applications, the complete information of $r(i,j)$ can not always be available to the controller. Hence, in this paper, the hidden Markov model $(r(i,j),\sigma(i,j),\varLambda,\varPsi)$ as in [refto] is introduced to characterize the asynchronous phenomenon between the controller and the system. The parameter $\sigma(i,j)$, refers to controller mode, takes values in another finite set $\mathcal{N}_{2} = \{1,2...N_{2}\}$, and satisfies the conditional probability matrix $\varPsi=\{\mu_{ks}\}$ with conditional mode transition probabilities
	\begin{equation}
		\Pr\{\sigma(i,j)=s|r(i,j)=k\}=\mu_{ks} %\forall  k\in\mathcal{N}_{1}, s\in\mathcal{N}_{2}
	\end{equation}
	where $\mu_{ks}\in[0,1]$ for all $k\in\mathcal{N}_{1}, s\in\mathcal{N}_{2}$, and $\sum_{s=1}^{N2}\mu_{ks} = 1$ for any mode $k$.
	
	Next, the definitions of  asymptotically mean square stable and $H_{\infty}$ performance for 2D systems will be given in Definition \ref{mean-square-stable} and Definition \ref{H_infty-performance}, respectively.
	
	\begin{definition}\label{mean-square-stable}
	The 2D Markov jump system \eqref{system-equation} with $w(i,j)\equiv0$ is said to be asymptotically mean square stable if the following holds:
	\begin{equation}\label{AMSS}
			\lim\limits_{i+j\to\infty}\mathbb{E}\{\|x(i,j)\|^{2}\} = 0
	\end{equation}
	for any boundary condition $X_{0}$ with Assumption \ref{boundary-assumptin}.
	\end{definition}

%	\begin{definition}\label{mean-square-stable}
%	If for $w(i,j)\equiv0$, boundary condition $(X_{0},\varGamma_{0})$, the following formulation holds:
%	\begin{equation}
%	\lim\limits_{i+j\to\infty}\mathbb{E}\{\|x(i,j)\|^{2}\} = 0
%	\end{equation}
%	then the 2D Markov jump system \eqref{system-equation} is said to be asymptotically mean square stable.
%	\end{definition}

	\begin{definition}\label{H_infty-performance}
		Given a scalar $\gamma>0$, the 2D Markov jump system \eqref{system-equation} is said to be asymptotically mean square stable with an $H_{\infty}$ disturbance attenuation performance $\gamma$ if the system satisfies \eqref{AMSS}, and under zero boundary condition, the following holds:
		\begin{equation} \label{invDef2}
			\sum_{i=0}^{\infty}\sum_{j=0}^{\infty}\mathbb\{\|y(i,j)\|^{2}\} <  \gamma^{2} \sum_{i=0}^{\infty}\sum_{j=0}^{\infty}\mathbb\{\|w(i,j)\|^{2}\}
		\end{equation}
		for all $w(i,j)\in\ell_{2}\{[0,\infty),[0,\infty)\}$.
	\end{definition}
	
	Now, we will make some notational simplification for convenience. The parameter $r(i,j)$ is represented by $k$, $r(i+1,j)$ and $r(i,j+1)$ are represented by $\tau$, $\sigma(i,j)$ is represented by $s$. 
	
	
	
	 The objective of this work is to devise an asynchronous  SMC law $u(i,j)$, such that the 2D Markov jump system \eqref{system-equation} is asymptotically mean square stable with an $H_{\infty}$ disturbance attenuation performance $\gamma$. 

\section{Main Result}

\subsection{ Sliding surface and sliding mode controller} \label{sliding-surface}
	In this paper, a novel Two-dimensional sliding surface function is constructed as follows: 
	\begin{equation}\label{siding-surface-equation}	
		s(i,j) = \begin{bmatrix}
					s^{h}(i,j)\\
					s^{v}(i,j)
					\end{bmatrix}
			   = Gx(i,j)
	\end{equation}
	where $G=\sum_{k=1}^{N_{1}}\beta_{k}G^{T}_{k}$, and scalars $\beta_{k}$ should be chosen such that $GB_{k}$ is nonsingular for any $k\in\mathcal{N}_{1}$. Based on the the assumption that $B_{k}$ is full column rank for any $k\in\mathcal{N}_{1}$, we can find that the above condition can be guaranteed easily with the properly selected parameter $\beta_{k}$. 
%	by selecting parameter $\beta_{k}$ properly. 
 	
 	An asynchronous 2D-SMC law is designed as follows:
	\begin{equation}\label{smc-law}
		u(i,j) = K_{s}x(i,j)-\rho(i,j)\frac{s(i,j)}{\|s(i,j)\|}
	\end{equation}
	for any $s\in\mathcal{N}_{2}$, where the matrix $K_{s}\in\mathbb{R}^{n_u\times n_x}$ with $n_x=n_h+n_v$ will be determined later, and the parameter $\rho(i,j)$ is given as
	\begin{equation}
	\rho(i,j) = \varrho_{1}\|x(i,j)\| + \varrho_{2}\|w(i,j)\|
	\end{equation}
	with $\varrho_{1}=\max_{k\in\mathcal{N}_{1}} \{\delta_{k} \}$, $\varrho_{2} = \max_{k\in\mathcal{N}_{1}}\{\|(GB_{k})^{-1}GE_{k}\| \} $, and the parameter $\delta_{k}$ is given in \eqref{nonlinear-func}. 
	
	Combining system \eqref{system-equation} and the asynchronous 2D-SMC low \eqref{siding-surface-equation}, the closed-loop 2D markov jump system can be obtained easily as follows:
%	\begin{equation} \label{closed-loop-system-equation}
%	\left\{
%	\begin{array}{lr}
%	\begin{split}
%	\emph{\textbf{x(i, j)}} = \bar{A}_{ks}x(i,j)+B_{k}\bar{\rho}_{k}(i,j)+E_{k}w(i,j)
%	\end{split}\\
%	\begin{split}
%	y(i,j) = C_{k}x(i,j)+D_{k}w(i,j)
%	\end{split}
%	\end{array}
%	\right.
%	\end{equation}
	\begin{equation} \label{closed-loop-system-equation}
	\emph{\textbf{x(i, j)}} = \bar{A}_{ks}x(i,j)+B_{k}\bar{\rho}_{k}(i,j)+E_{k}w(i,j)
	\end{equation}
	where $\bar{A}_{ks} = A_{k}+B_{k}K_{s}$, and $\bar{\rho}_{k}(i,j)$ as follows
	\begin{equation*}
	\bar\rho_{k}(i,j)=f_{k}(x(i,j))-(\varrho_{1}\|x(i,j)\|+\varrho_{2}\|w(i,j)\|)\cdot\frac{s(i,j)}{\|s(i,j)\|}.
	\end{equation*}
	Then, based on the properties of norm, the following condition can be deduced easily
	\begin{equation}\label{norm-rho-inequality}
	\|\bar{\rho}_{k}(i,j)\| \leq (\varrho_{1}+\delta_{k})\|x(i,j)\| + \varrho_{2}\|w(i,j)\| .
	\end{equation} 
		
\subsection{Analysis of Stability and $H_{\infty}$ attenuation performance } \label{stability&H_infty}
 In this subsection, we focus on the stability and $H_{\infty}$ attenuation performance analysis for the closed-loop 2D system \eqref{closed-loop-system-equation}. A sufficient condition will be derived to guarantee the considered system is  asymptotically mean square stable with an $H_{\infty}$ attenuation performance $\gamma$.
\begin{theorem}\label{theorem1}
	Consider the  Markov jump system \eqref{system-equation} under the Assumption \eqref{boundary-assumptin} and with the asynchronous 2D-SMC law \eqref{smc-law}. For a given scalar $\gamma>0$, if there exist matrices $K_{s}\in\mathbb{R}^{n_u\times n_x}$,  $R_{k}=\mathrm{diag}\{R^{h}_{k},R^{v}_{k}\}>0$,  $Q_{ks}>0$, $T_{ks}>0$ and scalars $\epsilon_{k}>0$, for any $k\in\mathcal{N}_{1}, s\in\mathcal{N}_{2}$,  such that the following inequalities hold: 
	\begin{equation}\label{T1C1}
	B^{T}_{k}  	\mathcal{R}_{k} B_{k} -\epsilon_{k}I \leq 0
	\end{equation},
	\begin{equation}\label{T1C2}
	\mathcal{A} +2\Big(\sum_{s=0}^{N_{2}}\mu_{ks} \mathrm{diag}\{Q_{ks}, T_{ks}\}\Big) < 0
	\end{equation},
	\begin{equation}\label{T1C3}
	\hat{A}^{T}_{ks}\mathcal{R}_{k}\hat{A}_{ks} - \mathrm{diag}\{Q_{ks}, T_{ks}\} < 0
	\end{equation}
	where
	\begin{equation*}
	\mathcal{A}=\begin{bmatrix}
	\varPi_{1} & \varPi_{3}\\
	*&\varPi_{2}
	\end{bmatrix}
	\end{equation*} with
	\begin{equation*} \label{varPi}
	\left\{
	\begin{array}{lr}
	\begin{split}
	\varPi_{1}&=-R_{k}+4(\delta_{k}+\varrho_{1})^{2}\epsilon_{k}I+C^{T}_{k}C_{k}\\
	\varPi_{2}&=-\gamma^{2}I+D^{T}_{k}D_{k}+4\varrho_{2}^{2}\epsilon_{k}I\\
	\varPi_{3}&= C_{k}^{T}D_{k}\\
	\end{split}
	\end{array}
	\right.
	\end{equation*}
	and $\mathcal{R}_{k}=\sum_{\tau=1}^{N1}\lambda_{k\tau}R_{\tau}$, $\hat{A}_{ks}=\begin{bmatrix}
	\bar{A}_{ks}& E_{k}
	\end{bmatrix}$, 
	then, the closed-loop system \eqref{closed-loop-system-equation} is asymptotically mean square stable with an $H_{\infty}$ disturbance attenuation performance $\gamma$.
\end{theorem}


\begin{proof}
	Let's start the proof with the stability of system. We select the Lyapunov candidate as $V_{1}(i,j) = x^{T}(i,j)R_{k}x(i,j)$, then, define
	\begin{equation}\label{VAR-DELTA-V1}
	\varDelta V_{1}(i,j) = \emph{\textbf{x(i, j)}}^{T}R_{\tau}\emph{\textbf{x(i, j)}} - x^{T}(i,j)R_{k}x(i,j)
	\end{equation}
	Based on the closed-loop system equation \eqref{closed-loop-system-equation} with $w(i,j)=0$, it is easy to find that
	\begin{equation}
	\begin{split}
	&\mathbb{E}\{\varDelta V_{1}(i,j) \}\\
	&=  \sum_{s=0}^{N_{2}}\mu_{ks}\Big\{\big[\bar{A}_{ks}x(i,j)+B_{k}\bar{\rho}_{k}(i,j)\big]^{T}\mathcal{R}_{k}\\
	&\times\big[\bar{A}_{ks}x(i,j)+B_{k}\bar{\rho}_{k}(i,j)\big]\Big\}\\
	&- x^{T}(i,j)R_{k}x(i,j) \\
	&\leq x^{T}(i,j) \Big\{2\big(\sum_{s=1}^{N_{2}}\mu_{ks}\bar{A}^{T}_{ks}(i,j)\mathcal{R}_{k}\bar{A}_{ks}\big)\Big\}x(i,j)\\ &+2\bar{\rho}^{T}_{k}(i,j)B^{T}_{k}\mathcal{R}_{k}B_{k}\bar{\rho}_{k}(i,j) \\
	&-  x^{T}(i,j)R_{k}x(i,j)
	\end{split}
	\end{equation}
	Recalling the conditions given in \eqref{norm-rho-inequality} and \eqref{T1C1}, the following inequality can be further obtained
	\begin{equation}\label{combine-one-1}
	\begin{split}
		\mathbb{E}\{\varDelta  V_{1}(i,j) \} \leq x^{T}(i,j)\mathcal{G}_{ks}x(i,j)
	\end{split}
	\end{equation}
	where $\mathcal{G}_{ks}= 2\big(\sum_{s=0}^{N_{2}}\mu_{ks}\bar{A}^{T}_{ks}\mathcal{R}_{k}\bar{A}_{ks}\big)
	+ 2\epsilon_{k}(\delta_{k}+\varrho_{1})^{2}I- R_{k}$.
	The following inequality can be deduced from \eqref{T1C2} based on the properties of  matrix quadratic
	\begin{equation}
	2\big(\sum_{s=1}^{N_{2}}\mu_{ks}Q_{ks}\big)+4\epsilon_{k}(\delta_{k}+\varrho_{1})^{2}I+C^{T}_{k}C_{k}-R_{k} < 0
	\end{equation}
	which will further deduce
	\begin{equation}\label{combine-one-2}
			2\big(\sum_{s=1}^{N_{2}}\mu_{ks}Q_{ks}\big)+2\epsilon_{k}(\delta_{k}+\varrho_{1})^{2}I-R_{k} < 0
	\end{equation}
	The following inequality can be  inferred directly from condition \eqref{T1C3} 
	\begin{equation}\label{combine-one-3}
		\bar{A}^{T}_{ks}\mathcal{R}_{k}\bar{A}_{ks}-Q_{ks} < 0
	\end{equation}
	Combine \eqref{combine-one-2} and \eqref{combine-one-3}, we can infer that $\mathcal{G}_{ks}<0$, which is equivalent to 
	\begin{equation}
		\mathcal{G}_{ks} \leq -\alpha I
	\end{equation}
	with scalar $\alpha>0$.
	Recalling \eqref{combine-one-1}, we can further infer that
	\begin{equation}\label{VleqAlpha}
		\mathbb{E}\{\varDelta V_{1}(i,j) \} \leq-\alpha \mathbb{E}\{\|x(i,j)\|^{2} \}
	\end{equation}
	Summing up on the both side of \eqref{VleqAlpha}, we have
	\begin{equation} \label{levE}
		\mathbb{E}\Big\{\sum_{i=0}^{\kappa_{1}}\sum_{j=0}^{\kappa_{2}}  \|x(i,j)\|^{2} \Big\} \leq -\frac{1}{\alpha} \mathbb{E}\Big\{\sum_{i=0}^{\kappa_{1}}\sum_{j=0}^{\kappa_{2}}  \varDelta V_{1}(i,j)  \Big\}
	\end{equation}
	where parameters $\kappa_{1}$, $\kappa_{2}$ are any positive integers. By substituting $\varDelta V_{1}$ and $R_{k}$ with \eqref{VAR-DELTA-V1} and $R_{k}=\mathrm{diag}\{R^{\rm{h}}_{k},R^{\rm{v}}_{k}\}$ respectively, we obtain
	\begin{equation} \label{Vhv}
		\begin{split}
			&\sum_{i=0}^{\kappa_{1}}\sum_{j=0}^{\kappa_{2}}  \varDelta V_{1}(i,j)\\&= \sum_{i=0}^{\kappa_{1}}\big\{V^{v}_{1}(i,\kappa_{2}+1) - V^{v}_{1}(i,0) \big\}\\
			&-  \sum_{j=0}^{\kappa_{2}}\big\{V^{h}_{1}(\kappa_{1}+1,j) - V^{h}_{1}(0,j) \big\}\\
			&\leq -\big( \sum_{i=0}^{\kappa_{1}}V^{v}_{1}(i,0) + \sum_{j=0}^{\kappa_{2}}V^{h}_{1}(0,j)\big) \\
%			&\leq -\beta \sum_{\ell=0}^{\infty} \big(  \|x^{v}(\ell,0)\|^{2} + \|x^{h}(0,\ell)\|^{2} \big)
		\end{split}
	\end{equation} 
	where $V_{1}^{h}(i,j)$ and $V_{1}^{v}(i,j)$ are defined as   
	\begin{equation*}
	\left\{
	\begin{array}{lr}
	\begin{split}
	V^{h}_{1}(i,j)=x^{hT}(i,j)R^{h}_{r(i,j)}x^{h}(i,j)\\
	V^{v}_{1}(i,j)=x^{vT}(i,j)R^{v}_{r(i,j)}x^{v}(i,j)
	\end{split}
	\end{array}
	\right.
	\end{equation*}
	Recalling the boundary condition in Assumption \ref{boundary-assumptin}, and let $\kappa_{1}$, $\kappa_{2}$ tend to infinity, it follows from \eqref{levE} and \eqref{Vhv} that
	\begin{equation}
		\begin{split}
		&\mathbb{E}\Big\{\sum_{i=0}^{\kappa_{1}}\sum_{j=0}^{\kappa_{2}}  \|x(i,j)\|^{2} \Big\} \\
		&\leq -\frac{\beta}{\alpha} \sum_{\ell=0}^{\infty} \big(  \|x^{v}(\ell,0)\|^{2} + \|x^{h}(0,\ell)\|^{2} \big)\\
		&<\infty
		\end{split}	
	\end{equation}
	where $\beta$ is the maximum eigenvalue of $R^{h}(0,\ell)$ and $R^{v}(\ell,0)$, for any $\ell=0,1,2...$, which implies that \eqref{AMSS} holds. Thus, the asymptotically mean square stable of the considered system is proved. 
	
	Next, let's focus on the $H_{\infty}$ attenuation performance under zero boundary condition. Based on the closed-loop system equation \eqref{closed-loop-system-equation}, it is easy to find that
	\begin{equation}\label{DETAV1}
	\begin{split}
	&\mathbb{E}\{\varDelta V_{1}(i,j) \}\\
	&=  \sum_{s=0}^{N_{2}}\mu_{ks}\Big\{\big[\bar{A}_{ks}x(i,j)+B_{k}\bar{\rho}_{k}(i,j)+E_{p}w(i,j)\big]^{T}\\
	&\times \mathcal{R}_{k}\big[\bar{A}_{ks}x(i,j)+B_{k}\bar{\rho}_{k}(i,j)+E_{p}w(i,j)\big]\Big\}\\
	&- x^{T}(i,j)R_{k}x(i,j) \\
	&\leq \hat{x}^{T}(i,j) \Big\{2\big(\sum_{s=1}^{N_{2}}\mu_{ks}\hat{A}^{T}_{ks}(i,j)\mathcal{R}_{k}\hat{A}_{ks}\big)\Big\}\hat{x}(i,j)\\ &+2\bar{\rho}^{T}_{k}(i,j)B^{T}_{k}\mathcal{R}_{k}B_{k}\bar{\rho}_{k}(i,j) \\
	&-  x^{T}(i,j)R_{k}x(i,j)\\
	\end{split}
	\end{equation}
%	where $\hat{x}(i,j)=\begin{bmatrix}
%		x(i,j)\\ w(i,j)
%	\end{bmatrix}$, $\hat{A}_{ks}(i,j)=\begin{bmatrix}
%		\bar{A}_{ks}&E_{k}
%	\end{bmatrix}$.
	where
	\begin{equation*}
		\hat{x}(i,j)=\begin{bmatrix}
		x(i,j)\\ w(i,j)
		\end{bmatrix},\ \hat{A}_{ks}(i,j)=\begin{bmatrix}
		\bar{A}_{ks}&E_{k}
		\end{bmatrix}
	\end{equation*}
	Notice that from \eqref{norm-rho-inequality} and \eqref{T1C1}, we have
	\begin{equation}\label{invRho}
		\begin{split}
		 	&\bar{\rho}^{T}_{k}(i,j)B^{T}_{k}\mathcal{R}_{k}B_{k}\bar{\rho}_{k}(i,j)\\
		 	&\leq 2\epsilon_{k}\big((\delta_{k}+\varrho_{1})^{2}\|x(i,j)\|^{2}+\varrho_{2}^{2}\|w(i,j)\|^{2} \big)
		\end{split}
	\end{equation} 
	The following condition can be deduced easily from \eqref{T1C2} and \eqref{T1C3}
	\begin{equation}\label{T1P4}
	\mathcal{\varXi}_{ks}<0
	\end{equation}	
	where $\mathcal{\varXi}_{ks} \equiv \mathcal{A} +2\sum_{s=1}^{N_{2}}\mu_{ks}\hat{A}^{T}_{ks}\mathcal{R}_{k}\hat{A}_{ks}$. 
	Recalling the system \eqref{system-equation}, and substituting \eqref{invRho} into \eqref{DETAV1} yields
	\begin{equation}\label{DETAV1ZW}
		\begin{split}
			&\mathbb{E}\{\varDelta V_{1}(i,j)+\|z(i,j)\|^{2}-\gamma^{2}\|w(i,j)\|^{2}  \}\\
			&\leq \hat{x}^{T}(i,j)\mathcal{\varXi}_{ks} \hat{x}(i,j)<0
		\end{split}
	\end{equation}
	Noting \eqref{Vhv} with the zero boundary condition, we can infer that
	\begin{equation} \label{Vhv2}
	\begin{split}
	&\sum_{i=0}^{\kappa_{1}}\sum_{j=0}^{\kappa_{2}}  \varDelta V_{1}(i,j)\\
	&=\sum_{i=0}^{\kappa_{1}}V^{v}_{1}(i,\kappa_{2}+1) + \sum_{j=0}^{\kappa_{2}}V^{h}_{1}(\kappa_{1}+1,j) \\
	&\geq 0 \qquad \forall \kappa_{1},\kappa_{2} = 1,2,3...
	\end{split}
	\end{equation}
	Then, we can further deduce from \eqref{DETAV1ZW} and \eqref{Vhv2} that 
	\begin{equation}\label{DETAV1ZW2}
	\begin{split}
	&\sum_{i=0}^{\infty}\sum_{j=0}^{\infty}  \mathbb{E}\{\|z(i,j)\|^{2}-\gamma^{2}\|w(i,j)\|^{2}  \}\\
	&\leq \sum_{i=0}^{\infty}\sum_{j=0}^{\infty}  \mathbb{E}\{\varDelta V_{1}(i,j)+\|z(i,j)\|^{2}-\gamma^{2}\|w(i,j)\|^{2}  \}  \\
	&< 0
	\end{split}
	\end{equation}
	which implies \eqref{invDef2} holds. And this completes the proof of Theorem \ref{theorem1}.
	
	
\end{proof}


\begin{remark}
	Remark.
\end{remark}


\subsection{Analysis of reachability}\label{minimization} 
	 The reachability of the  designed asynchronous 2D-SMC low for the closed-loop system \eqref{closed-loop-system-equation} will be discussed in this subsection. By using a stochastic Lyapunov method, we provide a sufficient condition which will confirm that the designed asynchronous 2D-SMC law \eqref{smc-law} can force the state trajectories of the closed-loop system \eqref{closed-loop-system-equation} into a time-varying sliding region around the specified 2D sliding surface \eqref{siding-surface-equation}.
	 

\begin{theorem}\label{theorem2}	
	Consider the closed-loop 2D Markov jump system \eqref{closed-loop-system-equation} with asynchronous 2D-SMC law \eqref{smc-law}. If there exists matrices $K_{s}\in\mathbb{R}^{n_u\times n_x}$, $R_{k}>0$, $F_{k}>0$, and  scalars $\epsilon_{k}>0$, for any $k\in\mathcal{N}_{1}, s\in\mathcal{N}_{2}$, such that the condition \eqref{T1C1} and the following inequality hold
	\begin{equation} \label{T2C1}
		2\sum_{s=1}^{N_{2}} \bar{A}^{T}_{ks}\big(\mathcal{R}_{k}+G^{T}\mathcal{F}_{k}G\big)\bar{A}_{ks}-R_{k} <0
	\end{equation}
	where $\mathcal{R}_{k}$ is defined in Theorem \ref{theorem1}, and $\mathcal{F}_{k}=\sum_{\tau=1}^{N_{1}}\lambda_{k\tau}F_{\tau}$. Then, the state trajectories of the considered closed-loop system will be driven into the following sliding region $\mathcal{O}$, around the predefined sliding surface \eqref{siding-surface-equation}:
	\begin{equation}\label{smc-region}
		\mathcal{O}\equiv\Big\{\|s(i,j)\|\leq\ \rho^{*}(i,j) \Big\}
	\end{equation} 
	where $\rho^{*}(i,j) = \mathrm{max}_{k\in\mathcal{N}_{1}}\sqrt{\hat{\rho}_{k}(i,j)/
	\lambda_{\mathrm{min}}(F_{k})}$ with
	\begin{equation*}
		 \begin{split}
		 	\hat{\rho}_{k}(i,j)&=4\big(\|E^{T}_{k}\mathcal{R}_{k}E_{k}\|+ \|E^{T}_{k}G^{T}\mathcal{R}_{k}GE_{k}\|\\
		 	&+2\varrho_{2}^{2}(\|B^{T}_{k}\mathcal{F}_{k}B_{k}\|+ \|B^{T}_{k}G^{T}\mathcal{F}_{k}GB_{k}\| )\big)\|w(i,j)\|^{2}\\
		 	&+8\big(\|B^{T}_{k}\mathcal{R}_{T}B_{k}\|+\|B^{T}_{k}G^{T}\mathcal{R}_{T}GB_{k}\|\big)\\ &\times(\varrho_{1}+\delta_{k})^{2}\|x(i,j)\|^{2}.
		 \end{split}
	\end{equation*}
	and $\lambda_{\mathrm{min}}(F_{k})$ here denotes the minimum eigenvalue of $F_{k}$.
\end{theorem} 

\begin{proof} 
	First, let's define $\emph{\textbf{s(i, j)}} = \begin{bmatrix}
	s^{h}(i+1,j)\\ s^{v}(i,j+1)
	\end{bmatrix}$,  it is easy find that $\emph{\textbf{s(i, j)}}=G\emph{\textbf{x(i, j)}}$.  Then, we select the Lyapunov candidate as
	\begin{equation}
		V(i,j)=V_{1}(i,j)+V_{2}(i,j)
	\end{equation}
	where $V_{1}(i,j)$ is defined in Theorem \ref{theorem1}, $V_{2}(i,j)=s^{T}(i,j)F_{k}s(i,j)$. Similar with the proof in Theorem \ref{theorem1}, it is easy to find that
	\begin{equation} \label{P2T1}
		\begin{split}
			&\mathbb{E}\{\varDelta V_{1}(i,j) \}\\
			&=\mathbb{E}\Big\{\emph{\textbf{x(i, j)}}^{T}R_{\tau}\emph{\textbf{x(i, j)}} - x^{T}(i,j)R_{k}x(i,j)\Big\}\\
			&=  \sum_{s=0}^{N_{2}}\mu_{ks}\Big\{\big[\bar{A}_{ks}x(i,j)+B_{k}\bar{\rho}_{k}(i,j)+E_{k}w(i,j)\big]^{T}\\
			&\times\mathcal{R}_{k} \big[ \bar{A}_{ks}x(i,j)+B_{k}\bar{\rho}_{k}(i,j)+E_{k}w(i,j)\big]\Big\}\\
			&-x^{T}(i,j)R_{k}x(i,j)\\
			&\leq 2x^{T}(i,j)\sum_{s=1}^{N_{2}}\mu_{ks}\bar{A}^{T}_{ks}\mathcal{R}_{k}\bar{A}_{ks}x(i,j) \\
			&+2\big[B_{k}\bar{\rho}_{k}(i,j)+E_{k}w(i,j)\big]^{T}\mathcal{R}_{k}\\
			&\times\big[B_{k}\bar{\rho}_{k}(i,j)+E_{k}w(i,j)\big]\\
			&-x^{T}(i,j)R_{k}x(i,j)\\
			&\leq 2x^{T}(i,j)\sum_{s=1}^{N_{2}}\mu_{ks}\bar{A}^{T}_{ks}\mathcal{R}_{k}\bar{A}_{ks}x(i,j) \\
			&+ \bar{\rho}^{T}_{k}(i,j)B^{T}_{k}\mathcal{R}_{k}B_{k}\bar{\rho}_{k}(i,j)\\
			&+w^{T}(i,j)E^{T}_{k}\mathcal{R}_{k}E_{k}w(i,j)\\
			&-x^{T}(i,j)R_{k}x(i,j)
%&\mathbb{E}\{\varDelta V_{1}(i,j) \}\\
%=&\mathbb{E}\Big\{\emph{\textbf{x(i, j)}}^{T}R_{\tau}\emph{\textbf{x(i, j)}} - x^{T}(i,j)R_{k}x(i,j)\Big\}\\
%=&  \sum_{s=0}^{N_{2}}\mu_{ks}\Big\{\big[\bar{A}_{ks}x(i,j)+B_{k}\bar{\rho}_{k}(i,j)+E_{k}w(i,j)\big]^{T}\\
%\times&\mathcal{R}_{k} \big[ \bar{A}_{ks}x(i,j)+B_{k}\bar{\rho}_{k}(i,j)+E_{k}w(i,j)\big]\Big\}\\
%-&x^{T}(i,j)R_{k}x(i,j)\\
%\leq& 2x^{T}(i,j)\sum_{s=1}^{K_{2}}\mu_{ks}\bar{A}^{T}_{ks}\mathcal{R}_{k}\bar{A}_{ks}x(i,j) \\
%+&2\big[B_{k}\bar{\rho}_{k}(i,j)+E_{k}w(i,j)\big]^{T}\mathcal{R}_{k}\\
%\times&\big[B_{k}\bar{\rho}_{k}(i,j)+E_{k}w(i,j)\big]\\
%-&x^{T}(i,j)R_{k}x(i,j)\\
%\leq& 2x^{T}(i,j)\sum_{s=1}^{K_{2}}\mu_{ks}\bar{A}^{T}_{ks}\mathcal{R}_{k}\bar{A}_{ks}x(i,j) \\
%+& \bar{\rho}^{T}_{k}(i,j)B^{T}_{k}\mathcal{R}_{k}B_{k}\bar{\rho}_{k}(i,j)\\
%+&w^{T}(i,j)E^{T}_{k}\mathcal{R}_{k}E_{k}w(i,j)\\
%-&x^{T}(i,j)R_{k}x(i,j)
		\end{split}
	\end{equation}
	Along with the sliding function in \eqref{siding-surface-equation}, we have
	\begin{equation} \label{P2T2}
		\begin{split}
			&\mathbb{E}\{\varDelta V_{2}(i,j) \}\\
			&=\mathbb{E}\Big\{\emph{\textbf{s(i, j)}}^{T}F_{\tau}\emph{\textbf{s(i, j)}} - s^{T}(i,j)F_{k}s(i,j)\Big\}\\
			&=  \sum_{s=0}^{N_{2}}\mu_{ks}\Big\{\big[\bar{A}_{ks}x(i,j)+B_{k}\bar{\rho}_{k}(i,j)+E_{k}w(i,j)\big]^{T}\\
			&\times G^{T}\mathcal{F}_{k} G\big[ \bar{A}_{ks}x(i,j)+B_{k}\bar{\rho}_{k}(i,j)+E_{k}w(i,j)\big]\Big\}\\
			&-s^{T}(i,j)F_{k}s(i,j)\\
			&\leq 2x^{T}(i,j)\sum_{s=1}^{N_{2}}\mu_{ks}\bar{A}^{T}_{ks}G^{T}\mathcal{F}_{k}G\bar{A}_{ks}x(i,j) \\
			&+2\big[B_{k}\bar{\rho}_{k}(i,j)+E_{k}w(i,j)\big]^{T}G^{T}\mathcal{F}_{k}\\
			&\times G\big[B_{k}\bar{\rho}_{k}(i,j)+E_{k}w(i,j)\big]\\
			&-s^{T}(i,j)F_{k}s(i,j)\\
			&\leq 2x^{T}(i,j)\sum_{s=1}^{N_{2}}\mu_{ks}\bar{A}^{T}_{ks}G^{T}\mathcal{F}_{k}G\bar{A}_{ks}x(i,j) \\
			&+ \bar{\rho}^{T}_{k}(i,j)B^{T}_{k}G^{T}\mathcal{F}_{k}GB_{k}\bar{\rho}_{k}(i,j)\\
			&+w^{T}(i,j)E^{T}_{k}G^{T}\mathcal{F}_{k}GE_{k}w(i,j)\\
			&-s^{T}(i,j)F_{k}s(i,j)
		\end{split}
	\end{equation}
	Combing \eqref{P2T1} and \eqref{P2T2}, we can infer that 
	\begin{equation}\label{P2T3}
		\begin{split}
				&\mathbb{E}\{\varDelta V(i,j) \}\\
				&= \mathbb{E}\Big\{\varDelta_{1}(i,j)+\varDelta V_{2}(i,j) \Big\}\\
				&\leq x^{T}(i,j)\Big\{ 2\sum_{s=1}^{N_{2}}\mu_{ks}\bar{A}^{T}_{ks}\big(\mathcal{R}_{k} +G^{T}\mathcal{F}_{k}G\big)\bar{A}_{ks} \Big\} x(i,j) \\
				& +\vec{\rho}_{k}(i,j)-x^{T}(i,j)R_{k}x(i,j) - \lambda_{\mathrm{min}}(F_{k})\|s(i,j)\|^{2}\\
		\end{split}
	\end{equation}
	where
	\begin{equation*}
		\begin{split}
			&\vec{\rho}_{k}(i,j)\\
			&= 4\big(\|B^{T}_{k}\mathcal{F}_{k}B_{k}\|+ \|B^{T}_{k}G^{T}\mathcal{F}_{k}GB_{k}\|\big) \|\bar{\rho}_{k}(i,j)\|^{2} \\
			&+ 4\big(\|E^{T}_{k}\mathcal{R}_{k}E_{k}\|+ \|E^{T}_{k}G^{T}\mathcal{R}_{k}GE_{k}\|\big) \|w(i,j)\|^{2} 
		\end{split}
	\end{equation*}
	Recalling the condition \eqref{norm-rho-inequality}, we can get an inequality as follows
	\begin{equation}\label{rholeseq}
		\|\bar{\rho}_{k}(i,j)\|^{2} \leq 2(\varrho_{1}+\delta_{k})^{2}\|x(i,j)\|^{2} +2\varrho_{2}^{2}\|w(i,j\|^{2} 
	\end{equation}	  
	It is obvious that $\vec{\rho}_{k}(i,j) < \hat{\rho}_{k}(i,j) $ for any $k\in\mathcal{N}_{1}$ after substitute \eqref{rholeseq} into $\vec{\rho}_{k}(i,j)$. 
	Then, based on the condition \eqref{T1C2}, when the state trajectories is out of the region $\mathcal{O}$ around the specified siding surface \eqref{siding-surface-equation}, we can infer that
	\begin{equation}\label{P2T5}
		-\lambda_{\mathrm{min}}(F_{k})\|s(i,j)\|^{2} + \vec{\rho}_{k}(i,j) <0
	\end{equation}
	It yields from \eqref{T2C1}, \eqref{P2T1}, \eqref{P2T3} and \eqref{P2T5} that 
	\begin{equation}
		\begin{split}
			&\mathbb{E}\{\varDelta V(i,j) \}\\
			&\leq x^{T}(i,j)\Big\{ 2\sum_{s=1}^{N_{2}}\mu_{ks}\bar{A}^{T}_{ks}\big(\mathcal{R}_{k} +G^{T}\mathcal{F}_{k}G \big)\bar{A}_{ks}\\ &-R_{k} \Big\} x(i,j)<0
		\end{split}
	\end{equation}
	which means the state trajectories of the close-loop \eqref{closed-loop-system-equation} are strictly decreasing (with mean square) outside the region $\mathcal{O}$ defined in \eqref{smc-region}. Now, the proof is complete.
	 
\end{proof}

\begin{remark}
	Remark.
\end{remark}


 
\subsection{  Synthesis of Asynchronous 2D-SMC Law }\label{smc-law-synthesis} 
It is obvious that, if the theorem \ref{theorem1} and the theorem \ref{theorem2} hold simultaneously, then, the asymptotically mean square stability  with an $H_{\infty}$ disturbance attenuation performance $\gamma$ of the closed-loop 2D system \eqref{closed-loop-system-equation} and the reachability of the predefined sliding function \eqref{siding-surface-equation} can be guaranteed simultaneously. That is, the to be determined matrix $K_{s}$ in 2D-SMC law \eqref{smc-law} should ensure that theorem \ref{theorem1}, \ref{theorem2} are established at the same time. Now, in this subsection, we will continue our study with this idea.

\begin{theorem}\label{theorem3}
	Consider the  Markov jump system \eqref{system-equation} under the Assumption \eqref{boundary-assumptin} and with the asynchronous 2D-SMC law \eqref{smc-law}.  For a given scalar $\gamma>0$, if there exist matrices $\tilde{K}_{s}\in\mathbb{R}^{n_u\times n_x}$, $L_{s}\in\mathbb{R}^{n_x\times n_x}$, $\tilde{R}_{k}=\mathrm{diag}\{\tilde{R}^{h}_{k},\tilde{R}^{v}_{k}\}>0$, $\tilde{F}_{k}>0$,  $\tilde{Q}_{ks}>0$, $\tilde{T}_{ks}>0$ and scalars $\tilde{\epsilon}_{k}>0$, for any $k\in\mathcal{N}_{1}, s\in\mathcal{N}_{2}$,  such that the following inequalities hold: 
	\begin{equation} \label{T3C1}
		\begin{bmatrix}
			-\tilde{\epsilon}_{k}I &\mathscr{B}_{k}\\
			*& \mathscr{R}_{k}
		\end{bmatrix} < 0
	\end{equation}
	
	\begin{equation} \label{T3C2}
		\begin{bmatrix}
			\mathscr{L}_{ks}&\mathscr{A}_{ks}&\mathscr{G}_{ks}\\
			*&\mathscr{R}_{k}&0\\
			*&*&\mathscr{F}_{k}\\
		\end{bmatrix}<0
	\end{equation}
	
	\begin{equation}\label{T3C3}
		\begin{bmatrix}
			\mathscr{H}_{k}&\mathscr{D}_{k}&\mathscr{P}_{ks}&\mathscr{Y}_{ks}\\
			*&\mathscr{I}_{k}&0&0\\
			*&*&\mathscr{Q}_{ks}&0\\
			*&*&*&\mathscr{T}_{ks}\\
		\end{bmatrix} <0
	\end{equation}
	where 
	\begin{equation}\notag
		\begin{split}
			\mathscr{B}_{k}&= \begin{bmatrix}
				\sqrt{\lambda_{k1}}\tilde{\epsilon}_{k}B^{T}_{k}&\sqrt{\lambda_{k2}}\tilde{\epsilon}_{k}B^{T}_{k}&\cdots&\sqrt{\lambda_{kN_{1}}}\tilde{\epsilon}_{k}B^{T}_{k}
			\end{bmatrix}\\
			\mathscr{R}_{k}&= \mathrm{diag}\{-\tilde{R}_{1}, -\tilde{R}_{2}, \cdots, -\tilde{R}_{N_{1}} \} \\
			\mathscr{F}_{k}&= \mathrm{diag}\{-\tilde{F}_{1}, -\tilde{F}_{2}, \cdots, -\tilde{F}_{N_{1}} \} \\	
			\mathscr{I}_{k} &= \mathrm{diag}\{-\tilde{\epsilon}_{k}, -I, -I, -\tilde{\epsilon}_{k} \}\\	
			\mathscr{Q}_{ks}&= \mathrm{diag}\{-\tilde{Q}_{k1}, -\tilde{Q}_{k2}, \cdots, -\tilde{Q}_{kN_{2}} \}\\
			\mathscr{T}_{ks}&= \mathrm{diag}\{-\tilde{T}_{k1}, -\tilde{T}_{k2}, \cdots, -\tilde{T}_{kN_{2}} \}\\
			\mathscr{L}_{ks}&= \mathrm{diag}\{\tilde{Q}_{ps}-L^{T}_{s}-L_{s},  -\tilde{T}_{ks} \}\\
%			\begin{bmatrix}
%				\tilde{Q}_{ps}-L^{T}_{s}-L_{s}&0\\
%				*&-\tilde{T}_{ks}
%			\end{bmatrix}, \
			\mathscr{H}_{k} &= \begin{bmatrix}
				-\tilde{R}_{k} & \tilde{R}C^{T}_{k}D^{T}_{k}\\
				*& -\gamma^{2}I
			\end{bmatrix} \\
			\mathscr{A}_{ks} &=\begin{bmatrix}
				\sqrt{\lambda_{k1}} \tilde{A}^{T}_{ks}& \cdots&\sqrt{\lambda_{kN_{1}}} \tilde{A}^{T}_{ks}\\
				\sqrt{\lambda_{k1}}\tilde{T}_{ks}B^{T}_{k} &\cdots&\sqrt{\lambda_{kN_{1}}} \tilde{T}_{ks}B^{T}_{k}\\
			\end{bmatrix} \\
			\mathscr{G}_{ks} &=\begin{bmatrix}
				\sqrt{\lambda_{k1}} \tilde{A}^{T}_{ks}G^{T}& \cdots&\sqrt{\lambda_{kN_{1}}} \tilde{A}^{T}_{ks}G^{T}\\
				0  &\cdots& 0
			\end{bmatrix}\\
			\mathscr{D}_{ks} &= \begin{bmatrix}
				2(\varrho_{1}+ \delta_{k} )\tilde{R}_{k}& \tilde{R}_{k}C^{T}_{k}&0&0\\
				0&0&D^{T}_{k}&2\varrho_{2} \\
			\end{bmatrix} \\	
			\mathscr{P}_{ks} &= \begin{bmatrix}
				\sqrt{2\mu_{k1}} \tilde{R}_{k} &\cdots &\sqrt{2\mu_{kN_{2}}} \tilde{R}_{k}\\
				0 & \cdots & 0\\
			\end{bmatrix}	\\
			\mathscr{Y}_{ks} &= \begin{bmatrix}
				0 & \cdots & 0\\
				\sqrt{2\mu_{k1}} I &\cdots &\sqrt{2\mu_{kN_{2}}} I\\
			\end{bmatrix}	\\						
		\end{split}
	\end{equation}
	and $\tilde{A}_{ks} =  A_{k}L_{s} + B_{k}\tilde{K}_{s}$. Then,  the closed-loop system \eqref{closed-loop-system-equation} is asymptotically mean square stable with an $H_{\infty}$ disturbance attenuation performance $\gamma$, and the state trajectories of the considered closed-loop system will be driven into a sliding region $\mathcal{O}$, around the predefined sliding surface \eqref{siding-surface-equation}. Moreover, the to be determined matrix $K_{s}$ in 2D-SMC law \eqref{smc-law} can be chosen as 
	\begin{equation}
		K_{s} = \tilde{K}_{s}L^{-1}_{s}
	\end{equation}
	if the LMIs \eqref{T3C1}, \eqref{T3C2} and \eqref{T3C3} have feasible solutions.
\end{theorem}
\begin{proof}
	As we discussed above, the objective is to testify that, the conditions \eqref{T1C1}, \eqref{T1C2}, \eqref{T1C3} in Theorem \ref{theorem1} and \eqref{T2C1} in Theorem \ref{theorem2}  can be guaranteed simultaneously by \eqref{T3C1}, \eqref{T3C2}, \eqref{T3C3}. Before that, let's make some notations as $\tilde{K}_{s}=K_{s}L_{s}$, $\tilde{R}_{k}=R^{-1}_{k}$, $\tilde{F}_{k}=F^{-1}_{k}$, $\tilde{Q}_{ks}=Q^{-1}_{ks}$, $\tilde{T}_{ks}=T^{-1}_{ks}$ and  $\tilde{\epsilon}=\epsilon^{-1}_{k}$. 
	Firstly, we will prove that \eqref{T1C1} and \eqref{T3C1} are equivalent. Pre- and post- multiplying the inequalities given in \eqref{T3C1} by $\mathrm{diag}\{\epsilon_{k}I, I,I,\cdots,I \}$, respectively, and applying  Schur complement after that, then, we can see \eqref{T1C1} satisfied.  
	Next, we will verity that \eqref{T3C2} and \eqref{T3C3} are sufficient  to ensure \eqref{T1C2}, \eqref{T1C3} and \eqref{T2C1} hold.  Using $\mathrm{diag}\{R_{k}, I ,I,\cdots, I \}$  to pre- and post-multiply the inequality given in \eqref{T3C3}, and applying Schur complement after that, then we will have \eqref{T1C2} satisfied.  
	It follows from  \eqref{T3C2} that $\tilde{Q}_{ps}-L^{T}_{s}-L_{s}<0$, that is $L^{T}_{s}+L_{s} $ is positive definite, which guarantees that the matrix  $L_{s}$ is invertible. We can infer the following formulation based on $Q_{ps}>0$
	\begin{equation}
		(\tilde{Q}_{ps} - L_{s} )^{T}\tilde{Q}^{-1}_{ks}(\tilde{Q}_{ps} - L_{s} )\geq 0
	\end{equation}
	which means 
	\begin{equation}
		-L^{T}_{s}\tilde{Q}_{ks}L_{s} \leq  \tilde{Q}_{ps}-L^{T}_{s}-L_{s}
	\end{equation}
	 Noting the condition give in \eqref{T3C3}, we can infer that
	 \begin{equation} \label{ls2}
	 	\begin{bmatrix}
	 	\tilde{\mathscr{L}}_{ks}&\mathscr{A}_{ks}&\mathscr{G}_{ks}\\
	 	*&\mathscr{R}_{k}&0\\
	 	*&*&\mathscr{F}_{k}\\
	 	\end{bmatrix}<0
	 \end{equation}
	 where $\tilde{\mathscr{L}}_{ks}= \mathrm{diag}\{-L^{T}_{s}\tilde{Q}_{ks}L_{s},  -\tilde{T}_{ks} \}$. \\
	 Noting the slack matrix $L_{s}$ is invertible, we denote $h_{ks}= \mathrm{diag}\{L^{-1}_{s}, T_{ks}, I, I,\cdots, I \}$. Using $h_{ks}$ to  pre- and post-multiply the inequality given in \eqref{ls2}, and applying Schur complement after that, then the following inequality will be obtained
	 \begin{equation} \label{ls3}
	 	\hat{A}^{T}_{ks}\mathcal{R}_{k}\hat{A}_{ks} + \check{A}^{T}_{ks}\mathcal{F}_{k}\check{A}_{ks}  - \mathrm{diag}\{Q_{ks}, T_{ks}\} < 0  
	 \end{equation}
	 where $\check{A}_{ks}=\begin{bmatrix}
	 \bar{A}_{ks}&0
	 \end{bmatrix}$. Combing \eqref{T1C2} and \eqref{ls3} we have
	 \begin{equation}
	 	\mathcal{A} +2\sum_{s=0}^{N_{2}}\mu_{ks}\Big\{ \hat{A}^{T}_{ks}\mathcal{R}_{k}\hat{A}_{ks} + \check{A}^{T}_{ks}\mathcal{F}_{k}\check{A}_{ks} \Big\} < 0
	 \end{equation}
	 which further implies \eqref{T2C1} holds based on the property of positive definite matrix. It is clear that, the gain matrix $K_{s}$ can not obtained directly from LMIs in Theorem \ref{theorem3} while $\tilde{K}_{s}$ is obtained. Thanks to the matrix $L_{s}$ is invertible, $K_{s}$ can be calculated indirectly with $K_{s}=\tilde{K}_{s}L^{-1}_{s}$. Now, the proof is finished.
\end{proof}



















\section{Numerical Example} \label{example}
In this section, we provide an example to verify the validity of the proposed method. 





\section{Conclusions} \label{conclusion}



% conference papers do not normally have an appendix

% use section* for acknowledgment
%\section*{Acknowledgment}
%
%
%The authors would like to thank...


\begin{thebibliography}{1}
	
	\bibitem{zhang-guangming}
	Zhang, Guangming, et al. "Finite-time H ∞ static output control of Markov jump systems with an auxiliary approach." Applied Mathematics \& Computation 273.C(2016):553-561.
	
\end{thebibliography}




% that's all folks
\end{document}





